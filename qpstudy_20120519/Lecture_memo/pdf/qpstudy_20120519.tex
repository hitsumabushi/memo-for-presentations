\documentclass{jsarticle}
\begin{document}
\section{2012/05/19 qpstudy}

\subsection{だれでもわかるハードウェア : Akihiro Kuwano}

大雑把な話をやります

\subsubsection{自己紹介}

\begin{itemize}
\item
  @kuwa\_tw
\end{itemize}
\subsubsection{コンピュータの5大要素}

\begin{itemize}
\item
  入力
\item
  演算
\item
  制御
\item
  記憶
\item
  出力
\end{itemize}
\subsubsection{ノイマン型コンピュータ}

\begin{itemize}
\item
  今のコンピュータの基礎
\end{itemize}
\subsubsection{コンピュータの基本構造}

\begin{itemize}
\item
  5大要素が基本!!
\end{itemize}
\paragraph{入力 : キーボード}

\begin{itemize}
\item
  ユーザからのデータ扱う
\end{itemize}
\paragraph{記憶 : HDD or SSD, メモリ}

\begin{itemize}
\item
  入力したデータをCPUに渡すためなど、置いておく。
\item
  補助記憶・主記憶
\end{itemize}
\paragraph{演算 : CPU}

\begin{itemize}
\item
  演算をする
\end{itemize}
\paragraph{制御 : CPU}

\begin{itemize}
\item
  各ハードウェアの動作を制御
\item
  CPU と一緒になっている
\end{itemize}
\paragraph{出力 : ディスプレイ}

\begin{itemize}
\item
  ユーザへの出力
\end{itemize}
\subsubsection{業務へ}

\begin{itemize}
\item
  ボトルネック調査
  \begin{itemize}
  \item
    CPU
  \item
    Memory
  \item
    Disk IO
  \item
    IO stat やらを見ます
  \end{itemize}
\item
  効率の良いプログラム
\item
  効率の良いアーキテクチャ
  \begin{itemize}
  \item
    効率が良いとは\ldots{}
    コンピュータにとって,アーキテクチャにとって良いということ
  \end{itemize}
\end{itemize}
\subsubsection{質疑応答}

\begin{itemize}
\item
  ENIAC : 最初のコンピュータっぽいやつ
  \begin{itemize}
  \item
    物理的な回路を組み替えまくってプログラミングする :
    ハードウェアコーディング
  \end{itemize}
\item
  トランジスタ : 切り替えスイッチだよー
  \begin{itemize}
  \item
    分岐の処理とか
  \item
    昔は真空管だったよ
  \end{itemize}
\item
  ボトルネック調査の具体例とかは?
  \begin{itemize}
  \item
    入出力はコンピュータの性能とあまり関係がない
  \item
    単体を調べればいい場合 : CPU, Memory, Disk
  \item
    アーキテクチャを考える場合 : CPU と主記憶をつなぐ部分,
    ディスクをつなぐバス部分 \ldots{}
  \end{itemize}
\end{itemize}
\subsection{CPU の話 : sho kisaragi}

CPU の動作を理解することで問題解析に役立てる

\subsubsection{自己紹介}

\begin{itemize}
\item
  @sho7650
\item
  カレー・パヒューム・深田恭子
\item
  CE -\textgreater{} ITスペシャリスト -\textgreater{} ITアーキテクト
\end{itemize}
\subsubsection{目的}

\begin{itemize}
\item
  CPU の振るまいと原理
\item
  問題の解析
\end{itemize}
\subsubsection{コンピュータ}

\begin{itemize}
\item
  5大要素
\item
  ノイマン型アーキテクチャ : プログラム内蔵型(?)
\item
  Processor
\item
  Programs
\item
  Data
\end{itemize}
\subsubsection{CPU}

\begin{itemize}
\item
  CPU からの IO
  \begin{itemize}
  \item
    I/O
  \item
    Memory
  \end{itemize}
\end{itemize}
\subsubsection{CPU の動作 : 4つの原則}

\begin{itemize}
\item
  Fetch
  \begin{itemize}
  \item
    主記憶(Memory) から Data をとってくる
  \end{itemize}
\item
  Decode
  \begin{itemize}
  \item
    主記憶から取ってきた命令を回路で実行できる命令に変換する
  \end{itemize}
\item
  Execute
  \begin{itemize}
  \item
    変換した命令を実行する
  \end{itemize}
\item
  Write back
  \begin{itemize}
  \item
    (ものによって違うどこに返すかが違うが)出力を返す
  \end{itemize}
\end{itemize}
\subsubsection{CPU Clock}

\begin{itemize}
\item
  水晶(クオーツ)に電圧をかける
\item
  1 Clock ごとに動作が行われる
\item
  世界最速は 5.2 GHz : IBM
\end{itemize}
\paragraph{Pipeline}

\begin{itemize}
\item
  1 Clockごとに、それぞれ1つ(Fetch〜Write back)を行う
\item
  Super Pipeline : fetch を 1 Clock
\end{itemize}
\paragraph{Super Scalar}

\begin{itemize}
\item
  例えば、計算が全部足し算だったら一気にやればいい。
\end{itemize}
\paragraph{Out-of-Order}

\begin{itemize}
\item
  処理に時間かかるものと、すぐ終わるものの順番を整理して、トータルで早くする。\\
  例えば、5
  個のタスクがあって、2番目だけがメモリ読み出しだったとする。\\
  メモリの内容が、5番目まで使われないなら、並列に2と3,4番目のタスクをやれば良い。
\end{itemize}
\subsubsection{CISC vs RISC}

\begin{itemize}
\item
  CISC : 回路を大きくする。命令豊富で何でもできるけど、1
  Clockでできることが減る。
\item
  RISC : 回路を小さくする。回路は単純で、命令を組み合わせて、トータルで1
  Clockになるくらい。
\item
  中はRISC で動いてるんだけど、外からの命令は CISC のように豊富にしてる?
\end{itemize}
\subsubsection{Peak CPU Clocks}

\begin{itemize}
\item
  計算速度をあげる
  \begin{itemize}
  \item
    光の速さ
  \item
    高電圧をかける -\textgreater{} 回路中で干渉して、計算するのが大変!!
  \end{itemize}
\end{itemize}
\subsubsection{CPU がどんな風に動くか}

\begin{itemize}
\item
  Z80 : 8 bit CPU で学んでみる
\end{itemize}
\paragraph{What is bit width?}

\begin{itemize}
\item
  あんまり定義がない
\item
  1回の命令が 8bit とか。
\item
  命令セット 16 bit とかはありえる。
\item
  bit 数が増える -\textgreater{} 1回の命令で伝える内容が増える
  -\textgreater{} 早くなるハズ
\end{itemize}
\paragraph{Machine Lang.}

やっぱりマシン語を学ぼう\\* Assembler Lang.

\paragraph{Register vs Memory}

Register をどう使うか!? -\textgreater{} Assembler をやればわかる

\begin{itemize}
\item
  PC(Program Counter) が実行してる箇所
\end{itemize}
\paragraph{Endian}

\begin{itemize}
\item
  Big Endian : 順(``ABCD'' を ABCD でメモリに格納)
\item
  Little Endian : 普通の順(``ABCD'' を CDAB でメモリに格納)
\item
  とにかく、CPU によって格納のされ方がちがう
\end{itemize}
\paragraph{Aレジスタ (Accumulator)}

\begin{itemize}
\item
  必ずここを経由して計算を行う \#\#\#\# Status (flug) Register
\item
  計算の結果を反映する
\item
  計算結果で繰り上がりがあったら、それを格納したり、
\item
  0 になったら、0 ということを格納する。
\end{itemize}
\subsubsection{Functions}

\begin{itemize}
\item
  Data Transfer
  \begin{itemize}
  \item
    Data Transfer(LD, PUSH, POP)
  \item
    Exchange
  \item
    Block Transfer
  \end{itemize}
\item
  Data Processing
  \begin{itemize}
  \item
    Arithmetric Operation(ADD,SUB,INC,DEC)
  \item
    Logical(AND,XOR,OR,CP)
  \item
    Skew(RL,RR,SLA,SRA)
  \end{itemize}
\item
  Test and Jump
  \begin{itemize}
  \item
    Jump(JP,JR,DZNJ,CALL,RET)
  \item
    TEST の結果はflagに入ってその結果を見て、Jumpする。
  \end{itemize}
\item
  Input/Output
  \begin{itemize}
  \item
    Input
  \item
    Output
  \end{itemize}
\item
  Control
  \begin{itemize}
  \item
    NOP,HALT
  \end{itemize}
\end{itemize}
\subsubsection{Mnemonics and Operands}

ex. Hello World

\textbar{} Mnemonic \textbar{} Operand \textbar{} LD \textbar{}
DE,0C000H \textbar{} LD \textbar{} HL,\#MSG LOOP: \textbar{} LD
\textbar{} A,(HL) \textbar{} OR \textbar{} A \textbar{} RET \textbar{} Z
\textbar{} LDI \textbar{} \textbar{} JR \textbar{} LOOP \ldots{}

\subsubsection{Summary}

\begin{itemize}
\item
  Assembler
\item
  English
\item
  いくつか違ってるものがあるかも。また、解釈の差があるかもしれません。
\item
  今日出たキーワードにつてい 自分で調べてください。
\end{itemize}
\subsection{コンピュータアーキテクチャ I/O 入門 : hasegawa さん}

\subsubsection{自己紹介}

\begin{itemize}
\item
  @hasegaw
\item
  Xen/KVM, FreeBSD virtio など
\item
  著書: いっぱい
\end{itemize}
\subsubsection{目標}

\begin{itemize}
\item
  コンピュータにおける I/O
\item
  IA-32 での I/O の種類
\item
  入出力デバイスの種類など
\end{itemize}
\subsubsection{入出力とは何か}

\begin{itemize}
\item
  I/O はコンピュータ(計算をするため)にとっては必須ではない
\end{itemize}
\subsubsection{入出力の基本}

\begin{itemize}
\item
  I/O ポート
\item
  割り込み
\item
  Direct Memory Access
\item
  メモリマップド I/O
\end{itemize}
\subsubsection{I/O ポート}

\begin{itemize}
\item
  コンピュータ出利用される古典的な入出力 例)
  ポート60番にデータを送ると、その先のデバイスにデータが届く
  ポート60番からインプットとかもできる。
\item
  プロセッサから外部に接続されるためのデジタルインターフェース
\item
  1 アドレス 8 ビットx64K, ON(1) or OFF(o) をつう有心
\item
  基本的に 8bit 単位なので低速
\item
  古いキーボード制御など
\end{itemize}
\subsubsection{割り込み}

\begin{itemize}
\item
  デバイスからCPUへイベントを通知するための信号
  \begin{itemize}
  \item
    IRQ 割り込み入力用の16本の信号線
  \end{itemize}
\item
  割り込みハンドラ
\item
  例えばキーボードの状態が変わると、イベントが起きて、割り込みが発生するなど。
\end{itemize}
\subsubsection{16 ページ(資料公開待ち)}

\begin{itemize}
\item
  int09\_handler : 割り込みハンドラ
  キーが押されてないかのチェックをしたりする。 \_int09\_function
  とかで読んでる?
\end{itemize}
\subsubsection{入出力装置をどのように使うのか}

\begin{itemize}
\item
  高校の頃?のライト点灯システムの実装例
\end{itemize}
\subsubsection{メモリマップドI/O(MMIO)}

\begin{itemize}
\item
  プロセッサの物理メモリ空間に、デバイス上のメモリ空間をマッピングする。
\item
  ソフトウェアからデバイス(普通のメモリに限らない)メモリにアクセスできる
\end{itemize}
\subsubsection{ダイレクトメモリアクセス(DMA)}

\begin{itemize}
\item
  プロセッサの指示にしたがってDMAコントローラがデータ転送する
\item
  データ転送の開始時/終了時のみプロセッサが干渉するため、プロセッサが時間を節約できる。(他の仕事もできる)
\end{itemize}
\subsubsection{IRQ と MSI}

\begin{itemize}
\item
  IRQ の数には制限がある
  \begin{itemize}
  \item
    インテルの現在のものだと、たかだか15本しかない
  \item
    他の目的にも使われる部分があるため、複数デバイスで共有するしかない。
  \item
    共有してるとどのデバイスが割り込みを受けたかわからないかもしれない
    -\textgreater{}
    デバイスがフラグを持っていて、それを使って区別している。
  \end{itemize}
\item
  キーボード、マウス、タイマ、\ldots{}.
\end{itemize}
\subsubsection{IRQ の共有}

\begin{itemize}
\item
  割り込みが発生したデバイスのフラグが 1 になる。
\item
  割り込みが発生したら各ドライバのステータスをチェックし、必要なハンドラのみが実行される。
\end{itemize}
\subsubsection{バスとはなにか}

\begin{itemize}
\item
  内部バス : CPU 内部
\item
  周辺回路 :
\item
  拡張バス : 外部デバイスとの接続
\end{itemize}
\subsubsection{PCI バス}

\begin{itemize}
\item
  コンピュータを各種ハードウェアデバイスに接続するための標準仕様
  \begin{itemize}
  \item
    32bit パラレル通信, 33MHz,帯域幅 133MB/s
  \end{itemize}
\item
  プラグアンドプレイ
  \begin{itemize}
  \item
    挿したら勝手に設定してくれる、とかいう方の意味
  \end{itemize}
\end{itemize}
\subsubsection{PCI-Xバス}

\begin{itemize}
\item
  PCI バスの上位互換
\item
  周波数とか、帯域がグレードアップ
\end{itemize}
\subsubsection{PCI Express}

\begin{itemize}
\item
  複数レーンを束ねて使う
\item
  シリアル転送になった!!!! \&\& スピードアップ!!!!
\item
  1 レーンで250MB/s
\item
  4 レーンで1 GB/s, 16 レーンで 4GB/s
\item
  活線挿抜に対応
\item
  ソフトウェアレベルでのPCI 上位互換
\end{itemize}
* * 25GHz の周波数でデータをシリアル伝送 (1Clock 当たり 2ビット伝送)

\subsubsection{PCI Express 2.0}

\begin{itemize}
\item
  クロック 2倍!
\item
  16 レーンで8GB/sの帯域をサポート
\item
  IRQ の代わりに\ldots{} Message Signal Interrupt のサポートが必須に!
\end{itemize}
\paragraph{skew とは}

\begin{itemize}
\item
  まず PCI Express は複数レーンを使っている。
\item
  各レーンごとに速度差があるかもしれないので、適当に同期を取る \&
  データ順を考慮して復元しないといけない
\end{itemize}
\subsubsection{Message Signal Interrupt(MSI) \textless{}-
この変更は仮想化環境に大きな影響があるに違いない!!パフォーマンスが上がる。}

\begin{itemize}
\item
  メモリ書き込みにより、割り込みを通知する
\item
  PCIデバイス利用する
\item
  IRQ を使わない
\item
  コントローラが仮想的に普通と違うメモリ書き込みをしたら、CPUがフラグをチェックせずとも、どのデバイスに割り込みが発生したかわかる!
\item
  32 メッセージまで共存可能
\item
  Enhanced MSI(MSI-X)
  \begin{itemize}
  \item
    PCI Express バス向けのMSI
  \item
    PCI Express 2.0 では実装必須
  \item
    2048 メッセージまで共存可能
  \end{itemize}
\end{itemize}
\subsubsection{CPU間のインターコネクト}

\begin{itemize}
\item
  マルチコア
\item
  マルチプロセッサ
  \begin{itemize}
  \item
    UMA : 対称型マルチプロセッサ
    \begin{itemize}
    \item
      どのCPUからでも同じ時間でアクセスできるが、全CPUでバスを共有する
    \end{itemize}
  \item
    NUMA : 非対称メモリアクセス
    \begin{itemize}
    \item
      近いデバイスには早くアクセスできるが、遠くのデバイスにはアクセスに時間がかかる
    \item
      バスの共有しない(?)か共有が少ないのか
    \item
      メモリとか増設するときにも、マルチプロセッサCPUだとどれかのCPUに近いやつとかあって気をつけよう。
    \item
      場合によっては遅くなる。チューニングを検討しよう
    \end{itemize}
  \end{itemize}
\item
  QPI (AMD)
\item
  HyperTransport (Intel)
\end{itemize}
\subsubsection{まとめ}

\begin{itemize}
\item
  I/O とは
  \begin{itemize}
  \item
    コンピュータのに対するデータ入出力の機能
  \end{itemize}
\item
  代表的なI/O方法
  \begin{itemize}
  \item
    I/Oポート, 割り込み
  \item
    メモリマップドI/O.DMA
  \end{itemize}
\item
  バス
  \begin{itemize}
  \item
    拡張バス
    \begin{itemize}
    \item
      PCI, PCI-X, PCI Express \ldots{}
    \end{itemize}
  \end{itemize}
\end{itemize}
\subsubsection{I/O を勉強するには}

\begin{itemize}
\item
  超初心者
  \begin{itemize}
  \item
    Arduino
  \item
    本もいっぱい, キットも買える
  \end{itemize}
\item
  初心者
  \begin{itemize}
  \item
    PIC
  \item
    USB マイコンボード とかも売ってる
  \end{itemize}
\item
  もっとガチな人
  \begin{itemize}
  \item
    BeagleBoard
  \end{itemize}
\end{itemize}
\subsection{インターフェース入門 shinonome}

\subsubsection{自己紹介}

\begin{itemize}
\item
  @H\_Shinonome
\end{itemize}
\subsubsection{資料が100 ページあるらしいので最初から挫折しておいた。}

\subsubsection{RS-232C}

\begin{itemize}
\item
  25 pin
\item
  9 pin のやつは標準規格に準拠してない。同期できないし。
\item
  同時送信数 : 1
\end{itemize}
\subsubsection{DB-60 : Cisco とかでよく見る}

\begin{itemize}
\item
  60 pin
\item
  115200 bps
\item
  同時送信数 : 1
\end{itemize}
\subsubsection{どっちも同時送信数 1}

\begin{itemize}
\item
  端子数多いのに\ldots{}
  \begin{itemize}
  \item
    直接データ送受信しない端子
  \item
    同時に同じデータを複数端子で送る
  \item
    そもそも使ってない端子もある
  \end{itemize}
\item
  信頼性の向上のため
\end{itemize}
\subsubsection{動作周波数}

\begin{itemize}
\item
  33MHz =\textgreater{} 1 cycle : 30 nano sec
\item
  SDR : Single Data Rate
\item
  DDR : Double Data Rate
\end{itemize}
\subsubsection{通信速度は頭打ち}

\begin{itemize}
\item
  費用対効果
  \begin{itemize}
  \item
    半導体性能
  \end{itemize}
\end{itemize}
\subsubsection{IEEE 1284}

\begin{itemize}
\item
  25 pin
\item
  プリンタポートとして出たものを拡張した規格
\item
  同時送受信 8 bit
\end{itemize}
\subsubsection{SCSI}

\begin{itemize}
\item
  50 pin
\item
  同時送受信 8 bit
\end{itemize}
\subsubsection{ATA}

\begin{itemize}
\item
  40 pin
\item
  同時送受信 16 bit
\end{itemize}
↑ パラレルが抜いた\ldots{}

\subsubsection{なぜシリアルに戻るのか}

\begin{itemize}
\item
  半導体性能アップ
\item
  スキュー
\end{itemize}
\subsubsection{USB (Universal Sirial Bus)}

\subsubsection{FireWire (IEEE 1394)}

\subsubsection{S-ATA}

\begin{itemize}
\item
  端子数: 7
\item
  通信速度: 1.5 Gbps - 6 Gbps
\end{itemize}
\subsubsection{SAS}

\begin{itemize}
\item
  端子数: 7
\item
  通信速度: 1.5 Gbps - 6 Gbps
\end{itemize}
\subsubsection{Thunderbolt}

\begin{itemize}
\item
  端子数: 20
\item
  通信速度: 10Gbps * 2
\item
  非同期パラレル通信
\end{itemize}
\subsubsection{InfiniBand}

\begin{itemize}
\item
  端子数: 4-48
\item
  通信速度: 2Gbps - 163.64 Gbps
\item
  シリアル(1X),非同期パラレル(2X - 12X)
\end{itemize}
\subsubsection{最近の流行 : 非同期パラレル}

\subsubsection{最後}

左: SAS 右: SATA

\begin{itemize}
\item
  違いを考えよう!!
\end{itemize}
\subsection{ニフティの宣伝}

\subsubsection{Nifty Cloud C4SA}

\begin{itemize}
\item
  5月末クローズドβ
\item
  APaaS + ブラウザ上のIDE + ソーシャル開発 的な感じ
\end{itemize}

\end{document}
