\documentclass{jsarticle}
\begin{document}
\section{2012/05/19:showstudy}

\begin{itemize}
\item
  プログラム初心者のためのお話
\end{itemize}
\subsection{ノイマン型コンピュータ}

\begin{itemize}
\item
  メモリ上の命令を下から順番に1つずつ実行する
\end{itemize}
\subsection{構造化プログラム とは}

\begin{itemize}
\item
  順次構造
\item
  分岐構造
\item
  反復構造
\end{itemize}
\subsection{サブルーチン の目的}

\begin{itemize}
\item
  再利用による効率化
\item
  バグの抑制
\end{itemize}
\subsection{オブジェクト指向}

\begin{itemize}
\item
  疎結合と継承
\item
  II効率化へ
\item
  属性とメソッド
\item
  API も粗結合\ldots{}
\end{itemize}
\subsection{アルゴリズム概論}

\begin{itemize}
\item
  1から40までカウントしていって、3の倍数または3のつく数字の時\ldots{}
  \begin{itemize}
  \item
    要求の分離
    \begin{itemize}
    \item
      1\textasciitilde{}40までカウント
    \item
      3の倍数のとき\ldots{}
    \item
      3のつく数字のとき\ldots{}
    \end{itemize}
  \end{itemize}
\item
  言語によらず、アルゴリズムは共通なので、まずそれを考えましょう。
\item
  代表的なアルゴリズム
  \begin{itemize}
  \item
    再帰的アルゴリズム
  \item
    ソート
  \item
    探索
  \item
    文字列探索
  \item
    データ圧縮
  \item
    暗号化アルゴリズム
  \item
    並列・分散アルゴリズム
  \item
    近似アルゴリズム・ヒューリスティックアルゴリズム
  \end{itemize}
\end{itemize}
\subsection{質疑応答}

\begin{itemize}
\item
  オブジェクト指向で大変なとこは?
  \begin{itemize}
  \item
    コメント書いたりしましょう
  \item
    フローチャートとかも大変な人もいるかもね
  \end{itemize}
\item
  インフラの人には何が大切なんですか?
  \begin{itemize}
  \item
    アプリの人が何に苦労していて、何を解消すればいいか、とか思いを共有したらいいと思う
  \item
    アプリの人とインフラの人の重点も違う。何を作ってるかも理解してやらないと失敗しますよ。
  \end{itemize}
\end{itemize}
\end{document}
